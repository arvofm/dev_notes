\documentclass{article}

\usepackage{parskip}
\usepackage[a4paper]{geometry}
\usepackage{mathtools}    % rearranges the equations display geometry

\begin{document}
    \sffamily

    \section*{Quadratic Equations}
    A quadratic equation looks like this:
    \begin{equation}    % numbered
        ax^2 + bx + c = 0. \label{eq:quad}
    \end{equation}
    where the variable x has two solutions as the discriminant root formula suggests:
    \begin{equation}
        \mathcal{X}_{1,2} = \frac{-b \pm \sqrt[2]{b^2 - 4ac}}{2a}. \label{eq:quadroot}
    \end{equation}
    If the \emph{discriminant} \(\Delta\) with \[ \Delta = b^2 - 4ac \] is zero, then
    the equation \ref{eq:quadroot} which stems from $\Delta$ becomes \[ x = \frac{-b}{2a} \]
    % \begin{math} ... \end{math} = \(...\) = $...$ ---------------------- inline
    % \begin{displaymath} ... \end{displaymath} = \[...\] = $$...$$ ------ displayed
    % \mathcal{X} or any uppercase letter is nice-looking
    % for text within formulas use \text{...}

    \section*{Weird Showcase}
    ddots: $\ddots$

    cdot: $\cdot$

    cdots: $\cdots$

    ldots: $\ldots$

    vdots: $\vdots$

    vectors: $\vec{a}+\vec b = \vec c$

    newton: $F\hat{i}=m \times a\hat{i}$

    \section*{Long Equations}
    Here we begin to use the mathtools package.
    Multiline equation:
    \begin{multline}
        \sum_{k=1}^n = 1^2 + 2^2 + 3^2 \\
        4^2 + 5^2 + 6^2 + \ldots \\
        (n-2)^2 + (n-1)^2 + n^2
    \end{multline}

    Gathered equation:
    \begin{gather}
        x+y+z = 0 \\
        y - z = 1
    \end{gather}

    To manipulate and align the equations where we want,
    \begin{align}
        a + b + c &= 5 \\
        b-c &= 1 \\
        a &= 2
    \end{align}
    % to use unnumbered equations, use * like align*

    \section*{Building Math Structures}
    Here is an array:
    \[
        A =  \left( 
            \begin{array}{cc}
                a_{11} & a_{12} \\
                a_{21} & a_{22}
            \end{array} 
        \right)
    \]

    Here is the binomial expansion:
    \[
        \binom{n}{k} = \frac{n!}{k!(n-k)!}
    \]

    Now some incorrect and cultural knowledge:
    \[
        N=\frac{-1}{12}=\underbrace{1+2+3+4+ \ldots}_{\infty}
    \]

\end{document}
