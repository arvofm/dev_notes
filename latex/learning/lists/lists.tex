\documentclass[12pt]{article}

\usepackage{paralist} % Compact lists, customizable labels and layout options
\usepackage{layouts} % Does all sorts of things about layouts

\begin{document}

    Here we go.

    \section*{Bulleted Lists}

    Latex provides handful of packages for designing the layout:
    \begin{itemize}
        \item Page Layout
            \begin{itemize}
                \item geometry
                \item typearea
            \end{itemize}
        \item Headers and Footers
            \begin{itemize}
                \item fancyhdr
                \item scrpage2
            \end{itemize}
        \item Line Spacing
            \begin{itemize}
                \item setspace
            \end{itemize}
    \end{itemize}

    \section*{Numbered Lists}
    Here is how you create an executable program:
    \begin{enumerate}
        \item Write the code.
        \item The compiler will compile the code for each file.
        \item The linker will link all the compiled code and create a solution file.
        \item The solution file be converted into an executable file by the maker.
    \end{enumerate}

    \section*{Description Lists}
    Some list related packages can be explained as,
    \begin{description}
        \item[paralist] provides compact lists, customizable label and layout options.
        \item[enumitem] gives control over labels and lengths in all kinds of lists.
        \item[desclist] offers more flexibility in descriptive lists.
        \item[multenum] produces vertical enumeration in multiple columns.
    \end{description}

    \section*{Saving Some Space}
    Here use the paralist package to provide a more compact view.
    \begin{compactenum}
        \item Write the code.
            \begin{compactitem} % We have compactenum, compactitem, compactdesc
            \item Your code should not contain any missing semicolons.
            \item Please write clean code.
            \end{compactitem}
        \item The compiler will compile the code.
            \begin{inparaenum} % We have inparaenum, inparaitem, inparadesc
            \item Compiler works for each file separately.
            \item It does not consider incompatibilities between the files.
            \end{inparaenum}
        \item The linker will create a solution file.
            \begin{asparaenum} % We have asparaenum, asparaitem, asparadesc
            \item But before, it will link all the compiled code.
            \item We can say that it is connecting the pieces.
            \end{asparaenum}
    \end{compactenum}

    \newpage

    \section*{Dimensions of a List}
    Here we have imported \textbf{layouts} package.
    \listdiagram
    You can freely customize all the environment variables.
    % \setlength{\labelwidth}{2cm}

\end{document}
