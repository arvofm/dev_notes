\documentclass{article}

\usepackage{parskip}
\usepackage{array}
\usepackage{booktabs}               % provides \toprule etc.
\usepackage{multirow}

\setlength{\extrarowheight}{5pt}    % use with array package

\begin{document}

    First, use tabbing to simulate a table.
    \begin{tabbing}
        \emph{Info:} \= Software \= : \= \LaTeX \\
        \> Author \> : \> Leslie Lamport \\
        \> Website \> : \> www.latex-project.org
    \end{tabbing}
    % \= sets a tab stop
    % \> goes to the next tab stop

    % \verb|code| typesets code as it is
    Real tables now.

    \newcommand{\head}[1]{\textnormal{\textbf{#1}}} % lets have head
    \begin{table}
    \centering
    \begin{tabular}{ccc}    % ccc standing for three centered columns
        \hline              % sep line
        \head{Name} & \head{Age} & \head{Job} \\      % & is for separating column entries
        \hline                          % horizontal line
        Ali & 20 & Student \\    % fuckers what is this
        \cline{2-3}                     % horizondtal line
        Daniel & 24 & Programmer \\
        Ale & 19 & Bass Player \\
    \end{tabular}
    \caption{Hello there}
    \end{table}

    Alignments here.

    \begin{tabular}{|l|c|r|p{2cm}|}     % | stands for vertical line
        \toprule[1.5pt]
        \head{left} & \head{centered} & \head{right} & \head{fully justified paragraph cell} \\
        \midrule
        l & casd \vline asd & r & p \\
        \bottomrule[1.5pt]
    \end{tabular}

    Another example.

    \begin{tabular}{c | p{2cm} | m{2cm} | b{2cm}}
        \hline
        baseline & aligned at the top & aligned at the middle & aligned at the bottom \\
        \hline
    \end{tabular}

    Another example.

    \begin{tabular}{@{} >{\itshape}l l !{:} l<{.} @{}}
        \hline
        \multirow{3}{*}{Info:} & Software & \LaTeX \\       % \multirow{rows}{width(* for auto)}{code}
        & Author & Miss Leslie \\
        & Website & www.latex-project.org \\
        & \multicolumn{2}{c<{.}}{An important person} \\
        \hline
    \end{tabular}

    % l         - left
    % c         - centered
    % r         - right
    % p{width}  - paragraph cell
    % @{code}   - insert code instead of empty space before or after a column
    % |         - stands for a vertical line
    % After loading \usepackage{array},
    % m{width}  - baseline is at the middle
    % p{width}  - baseline is at top
    % b{width}  - baseline is at bottom
    % !{code}   - like @{code}, but the space is not suppressed
    % >{code}   - inserts code at the beginning of each column, is used before lcrmasdasdad
    % <{code}   - inserts code at the end of each column, is used before lcrasdasd



\end{document}
